
% LaTeX Beamer file automatically generated from DocOnce
% https://github.com/hplgit/doconce

%-------------------- begin beamer-specific preamble ----------------------

\documentclass{beamer}

\usetheme{red_shadow}
\usecolortheme{default}

% turn off the almost invisible, yet disturbing, navigation symbols:
\setbeamertemplate{navigation symbols}{}

% Examples on customization:
%\usecolortheme[named=RawSienna]{structure}
%\usetheme[height=7mm]{Rochester}
%\setbeamerfont{frametitle}{family=\rmfamily,shape=\itshape}
%\setbeamertemplate{items}[ball]
%\setbeamertemplate{blocks}[rounded][shadow=true]
%\useoutertheme{infolines}
%
%\usefonttheme{}
%\useinntertheme{}
%
%\setbeameroption{show notes}
%\setbeameroption{show notes on second screen=right}

% fine for B/W printing:
%\usecolortheme{seahorse}

\usepackage{pgf}
\usepackage{graphicx}
\usepackage{epsfig}
\usepackage{relsize}

\usepackage{fancybox}  % make sure fancybox is loaded before fancyvrb

\usepackage{fancyvrb}
%\usepackage{minted} % requires pygments and latex -shell-escape filename
%\usepackage{anslistings}
%\usepackage{listingsutf8}

\usepackage{amsmath,amssymb,bm}
%\usepackage[latin1]{inputenc}
\usepackage[T1]{fontenc}
\usepackage[utf8]{inputenc}
\usepackage{colortbl}
\usepackage[english]{babel}
\usepackage{tikz}
\usepackage{framed}
% Use some nice templates
\beamertemplatetransparentcovereddynamic

% --- begin table of contents based on sections ---
% Delete this, if you do not want the table of contents to pop up at
% the beginning of each section:
% (Only section headings can enter the table of contents in Beamer
% slides generated from DocOnce source, while subsections are used
% for the title in ordinary slides.)
\AtBeginSection[]
{
  \begin{frame}<beamer>[plain]
  \frametitle{}
  %\frametitle{Outline}
  \tableofcontents[currentsection]
  \end{frame}
}
% --- end table of contents based on sections ---

% If you wish to uncover everything in a step-wise fashion, uncomment
% the following command:

%\beamerdefaultoverlayspecification{<+->}

\newcommand{\shortinlinecomment}[3]{\note{\textbf{#1}: #2}}
\newcommand{\longinlinecomment}[3]{\shortinlinecomment{#1}{#2}{#3}}

\definecolor{linkcolor}{rgb}{0,0,0.4}
\hypersetup{
    colorlinks=true,
    linkcolor=linkcolor,
    urlcolor=linkcolor,
    pdfmenubar=true,
    pdftoolbar=true,
    bookmarksdepth=3
    }
\setlength{\parskip}{7pt}  % {1em}

\newenvironment{doconceexercise}{}{}
\newcounter{doconceexercisecounter}
\newenvironment{doconce:movie}{}{}
\newcounter{doconce:movie:counter}

\newcommand{\subex}[1]{\noindent\textbf{#1}}  % for subexercises: a), b), etc

%-------------------- end beamer-specific preamble ----------------------

% Add user's preamble




% insert custom LaTeX commands...

\raggedbottom
\makeindex

%-------------------- end preamble ----------------------

\begin{document}

% matching end for #ifdef PREAMBLE

\newcommand{\exercisesection}[1]{\subsection*{#1}}

\input{newcommands}


% ------------------- main content ----------------------



% ----------------- title -------------------------

\title{Kort om kursene INF1100 og MAT-INF1100L}

% ----------------- author(s) -------------------------

\author{Hans Petter Langtangen\inst{1,2}}
\institute{Simula Research Laboratory\inst{1}
\and
University of Oslo, Dept.~of Informatics\inst{2}}
% ----------------- end author(s) -------------------------

\date{Aug 8, 2017
% <optional titlepage figure>
% <optional copyright>
}

\begin{frame}[plain,fragile]
\titlepage
\end{frame}

\begin{frame}[plain,fragile]
\frametitle{INF1100 er en første introduksjon til å programmere datamaskiner}

\begin{block}{}
\begin{itemize}
  \item Programmering er \emph{svært} viktig i industri og forskning!

  \item Programmering vil bli brukt i veldig mange senere emner - derfor er INF1100/MAT-INF1100L svært sentrale kurs

  \item Hvorfor?\\
    Programmeringen gjør matematikken mye mer anvendbar

  \item Tre perspektiver på matematikk i høst:
\begin{itemize}

    \item tradisjonell kalkulus (MAT1100/MAT1001)

    \item numerisk (datamaskinvennlig) matematikk (MAT-INF1100)

    \item programmering av numerisk matematikk (INF1100)
\end{itemize}

\noindent
\end{itemize}

\noindent
\end{block}
\end{frame}

\begin{frame}[plain,fragile]
\frametitle{MAT-INF1100L = INF1100 uke 1-6 + MAT-INF1100}

\begin{block}{}
\begin{itemize}
 \item Fullstendig sammenfallende undervisning og obliger med INF1100:
\begin{itemize}

   \item forelesninger: uke 34-39

   \item gruppeøvelser (obliger): uke 35-40

\end{itemize}

\noindent
 \item Samme midtveiseksamen som INF1100

 \item Fullstendig sammenfallende undervisning med MAT-INF1100 etter det
\end{itemize}

\noindent
\end{block}
\end{frame}

\begin{frame}[plain,fragile]
\frametitle{All informasjon og alle beskjeder ligger på nettsidene}

\begin{itemize}
 \item INF1100:
   \href{{http://www.uio.no/studier/emner/matnat/ifi/INF1100/h14}}{\nolinkurl{http://www.uio.no/studier/emner/matnat/ifi/INF1100/h14}}

 \item MAT-INF1100L:
   \href{{http://www.uio.no/studier/emner/matnat/math/MAT-INF1100L-h14}}{\nolinkurl{http://www.uio.no/studier/emner/matnat/math/MAT-INF1100L-h14}}

 \item Se spesielt \href{{https://www.uio.no/studier/emner/matnat/ifi/INF1100/h14/ressurser/undervisningsplan.html}}{INF1100 undervisningsplan} for info om hva som skjer hver uke
\end{itemize}

\noindent
\end{frame}

\begin{frame}[plain,fragile]
\frametitle{Undervisningen består av øvelser og forelesninger}

\begin{block}{}
\begin{itemize}
\pause
  \item Plenumsundervisning tirsdager og torsdager 14.15-16.00 i Sophus Lies auditorium

\pause
  \item 1. time: oppgaver fra forrige forelesningstime løses i plenum

\pause
  \item 2. time: forelesning av nytt stoff

\pause
  \item 2 t oppgaveløsning på terminalstue i mindre grupper der du kan få individuell veiledning

\pause
  \item Delta på \emph{alle} undervisningstimene!
\end{itemize}

\noindent
\end{block}
\end{frame}

\begin{frame}[plain,fragile]
\frametitle{Undervisningsmateriell}

\begin{columns}
\column{0.65\textwidth}
\begin{block}{}
\begin{itemize}
  \item Lærebok skrevet spesielt for INF1100

  \item Oppgavene foreligger som "PDF fil": "

  \item 2. time: forelesning av nytt stoff

  \item 2 t oppgaveløsning på terminalstue i mindre grupper der du kan få individuell veiledning

  \item Delta på \emph{alle} undervisningstimene!
\end{itemize}

\noindent
\end{block}

\column{0.35\textwidth}
% FIGURE: [http://hplgit.github.io/scipro-primer/figs/Primer4th_pic.jpg, width=350 frac=0.6]

\end{columns}
\end{frame}

\begin{frame}[plain,fragile]
\frametitle{Det kreves innlevering av 3-5 obligatoriske oppgaver hver uke}

\begin{block}{}
\begin{itemize}
  \item ``Løp 1'': Mange små obligatoriske oppgaver
\begin{itemize}

    \item 3-5 obligatoriske oppgaver hver uke\\
      (vurderes til bestått eller ikke bestått)

    \item De fleste oppgavene teller 1 poeng

    \item Krav INF1100: 15 (av 23) p før uke 41, + 20 (av 37) p før 1.~des.

    \item Krav MAT-INF1100L: 18 (av 23) p fra oppgavene i uke 35-39, men det blir gitt
      ekstraoppgaver etter midtveiseksamen

\end{itemize}

\noindent
  \item ``Løp 2'': Færre, men større obligatoriske oppgaver
\begin{itemize}

    \item Passer for dere med god programmeringserfaring

\end{itemize}

\noindent
  \item Eksamen:
\begin{itemize}

    \item Midtveiseksamen i uke 41 - teller 25\% av karakteren

    \item Avsluttende eksamen - teller 75\% av karakteren
\end{itemize}

\noindent
\end{itemize}

\noindent
\end{block}
\end{frame}

\begin{frame}[plain,fragile]
\frametitle{Hvordan du må jobbe}

\pause
\begin{block}{}
\begin{itemize}
  \item Foran hver forelesning må du ha lest ukens kapittel i læreboken

  \item Foran hver oppgaveløsning i plenum må du selv ha forsøkt å løse oppgavene (les kapittelet først!)

  \item Etterarbeid oppgavene når du har sett løsning i plenum

  \item Nå er du klar for ukens obliger: Du kan gjøre dem på terminalstue
    under veiledning
\end{itemize}

\noindent
\end{block}

\begin{block}{Merk: }
\begin{itemize}
  \item Spesielt forelesningene går frem mye fortere enn klasseromsundervisningen i videregående skole

  \item Undervisningen forutsetter at du er forberedt og at du kan forrige ukes temaer
\end{itemize}

\noindent
\end{block}
\end{frame}

\begin{frame}[plain,fragile]
\frametitle{Du må lære programmering ved å programmere mye}

\begin{block}{}
\begin{itemize}
  \item Du kan ikke lese deg til programmering

  \item De fleste synes programmering er krevende i begynnelsen - så blir det utrolig gøy!

  \item Oppskrift på suksess: vær godt forberedt til undervisningen - det gir deg mest fritid og mest læring

  \item Forventet arbeid er 13 timer med INF1100 hver uke \\
    (6 t undervisning, 7 t selvstudium)
\end{itemize}

\noindent
\end{block}
\end{frame}

\begin{frame}[plain,fragile]
\frametitle{Hvor mye matematikk må jeg kunne på forhånd?}

\begin{block}{}
\begin{itemize}
  \item Nesten alle eksemplene i INF1100 handler om bruk av matematikk

  \item Vi bygger (i prinsippet) på R2 fra vgs

  \item Men matematikken i INF1100 er stort sett \emph{numerisk} matematikk (MAT-INF1100)

  \item Vi håper at INF1100 skal belyse matematikk fra en ny vinkel og hjelpe deg til å forstå matematikk bedre samtidig som du lærer å programmere
\end{itemize}

\noindent
\end{block}
\end{frame}

\begin{frame}[plain,fragile]
\frametitle{Alt undervisningsmateriale er på engelsk}

\begin{block}{}
\begin{itemize}
  \item Muntlig undervisning foregår på norsk

  \item Alt skriftlig materiale er på engelsk

  \item Hvorfor?

  \item Det mangler gode norske ord for mange ord/uttrykk i programmering

  \item Du finner mye informasjon om programmering på nettet og i bøker - nesten all denne informasjonen er på engelsk og da må du kunne de engelske uttrykkene

  \item Mesteparten av undervisningsmateriellet på UiO er på engelsk

  \item I jobbsammenheng kan du regne med at alt skriftlig foregår på engelsk

  \item Boken og undervisningsmaterialet brukes ved mange utenlandske universiteter
\end{itemize}

\noindent
\end{block}
\end{frame}

\end{document}
